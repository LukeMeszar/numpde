\documentclass[12pt]{article}
\usepackage{amssymb, amsmath, amsthm, amsfonts}
\usepackage[alphabetic]{amsrefs}
\usepackage{mathrsfs,comment}
\usepackage{float}
\usepackage[all,arc,2cell]{xy}
\usepackage{enumerate}
\usepackage{tcolorbox}
\usepackage[margin=0.75in]{geometry}
\usepackage{MnSymbol}
\usepackage{pdfpages}
\usepackage{tikz}
\usetikzlibrary{arrows,automata,calc,positioning}
\usepackage[colorlinks = true,
linkcolor = blue,
urlcolor  = blue,
citecolor = blue,
anchorcolor = blue]{hyperref}

\usepackage{cleveref}


%theoremstyle{plain} --- default
\newtheorem{theorem}{Theorem}[section]
\newtheorem{corollary}[theorem]{Corollary}
\newtheorem{conjecture}[theorem]{Conjecture}
\newtheorem{proposition}[theorem]{Proposition}
\newtheorem{lemma}[theorem]{Lemma}

\theoremstyle{definition}
\newtheorem{definition}[theorem]{Definition}
\newtheorem{example}[theorem]{Example}
\newtheorem{remark}[theorem]{Remark}
\newtheorem*{solution}{Solution}


\makeatother
\numberwithin{equation}{section}



%%%Commands and shortcuts%%%%
%

\newcommand{\R}{\mathbb{R}}
\newcommand{\Q}{\mathbb{Q}}
\newcommand{\Z}{\mathbb{Z}}
\newcommand{\N}{\mathbb{N}}
\newcommand{\F}{\mathbb{F}}
\newcommand{\xb}{\mathbf{x}}
\newcommand{\yb}{\mathbf{y}}
\newcommand{\ab}{\mathbf{a}}
\newcommand{\zb}{\mathbf{z}}
\newcommand{\func}[3]{#1\colon #2 \rightarrow #3}
\newcommand{\sumn}{\sum_{i=1}^{n}}
\newcommand{\innerp}[2]{\langle #1,#2 \rangle}
\newcommand{\dsq}{d_{sq}}
\newcommand{\dtc}{d_{tc}}
\newcommand{\ifff}{if and only if }
\newcommand{\st}{such that }
\newcommand{\te}{there exists }
\newcommand{\wrt}{with respect to }
\newcommand{\wloG}{without loss of generality }
\newcommand{\Wlog}{Without loss of generality }
\newcommand{\ssteq}{\subseteq}
\newcommand{\setst}{\hspace{1mm} | \hspace{1mm} }
\newcommand{\Bs}{\mathscr{B}}
\newcommand{\Us}{\mathscr{U}}
\newcommand{\T}{\mathcal{T}}
\newcommand{\Y}{\mathcal{Y}}
\newcommand{\B}{\mathcal{B}}
\newcommand{\G}{\mathcal{G}}
\newcommand{\Pc}{\mathcal{P}}
\newcommand{\Sc}{\mathcal{S}}
\newcommand{\C}{\mathcal{C}}
\newcommand{\A}{\mathcal{A}}
\newcommand{\Lc}{\mathcal{L}}
\newcommand{\onen}{\{1,\ldots,n\}}
\newcommand{\finv}[2]{#1^{-1}\left(#2\right)}		
\newcommand{\es}{\varnothing}	
\newcommand{\ol}[1]{\overline{#1}}	
\newcommand{\xnn}{(\mathbf{x}_n)_{n \in \N}}
\newcommand{\sft}{\mathbf{SFT_1}}		
\newcommand{\cop}{\sqcup}	
\newcommand{\cyc}[1]{#1_1\ldots #1_n#1_1}	
\newcommand{\pcyc}[2]{#2(#1_1)\ldots #2(#1_n)#2(#1_1)}	
\newcommand{\cyci}[2]{#1_1\ldots #1_{#2}#1_1}
\newcommand{\pcyci}[3]{#3(#1_1)\ldots #3(#1_{#2})#3(#1_1)}
\newcommand{\xh}{\bar{x}}	
\newcommand{\yh}{\bar{y}}		
\newcommand{\equivcls}[1]{#1/{\sim}}	
\newcommand{\modab}[3]{#1 \equiv #2 \pmod{#3}}						
\newcommand{\farg}{{-}}																																						

\DeclareMathOperator{\Cl}{Cl}
\DeclareMathOperator{\Bd}{Bd}
\DeclareMathOperator{\Int}{Int}
\DeclareMathOperator{\id}{id}
\DeclareMathOperator{\Hom}{Hom}
\DeclareMathOperator{\Tor}{Tor}
\DeclareMathOperator{\Ext}{Ext}

\renewcommand{\epsilon}{\varepsilon}
%\renewcommand{\phi}{\varphi}
\renewcommand\qedsymbol{$\blacksquare$}

\usepackage{color}
\newcommand{\Comments}{1}
\newcommand{\raf}[1]{\ifnum\Comments=1\textcolor{blue}{[RF: #1]}\fi}


\title{CSCI 5636 Final Project Proposal \\
\large A finite element approach simulating flow over a Formula 1 car's wing using FEniCS}
\author{Luke Meszar}
\date{November 12, 2018}
\begin{document}
	\maketitle
	In this project, I want to simulate the flow over a Formula 1 car's wing. I will probably choose the rear wing since it is aerodynamically simpler. I will also choose to model it after older cars since the modern ones have such complicated aerodynamic features. If I can find the technical regulations from this era, I will look at them to see if I can make the geometry as accurately as possible. Also, this simulation will ignore a lot of the important real life factors. The biggest is the lack of a whole car simulation. The flow over the rear wing of a car depends entirely on how the rest of the car shapes the air flowing around it before it hits the rear wing. I will just be simulating the flow over the rear wing on its own.
	
	We will be looking at incompressible fluids governed by the Navier-Stokes equations. We let $\rho(x,t)$ be the density of the fluid at point $x$ at times $t$. Likewise, let $u(x,t)$ be its velocity. We start with the mass continuity equation
	\[\frac{\partial \rho}{\partial t}+\nabla \cdot (\rho u) = 0.\]
	We also consider the Cauchy momentum equation 
	\[\frac{\partial \rho u}{\partial t} + \nabla \cdot (pu \otimes u) + \nabla p - \eta \Delta u - \left(\frac{\eta}{3}+\xi\right)\nabla(\nabla\cdot u) = f\]
	where $f$ are the external body forces (gravity etc.) $p$ is the pressure and $\eta$ and $\xi$ are the first and second viscosities. Since we are concerned with incompressible flows where $\rho$ can be treated as a constant, then, we get the Navier-Stokes equations:
	\begin{gather*}
	\nabla \cdot u = 0 \\
	\frac{\partial u}{\partial t} + u\nabla u + \nabla p = f.
	\end{gather*}
	Using these equations, we will simulate how the air moves around a wing. I am note sure yet how to solve these equations with the geometry of the wing taken account of. Once I have results, I would also like to calculate the downforce (i.e. negative lift) and drag the wing generates. I may also compare the downforce and drag on two or three different design of wings. I will be using FEniCS as my finite element software for this project. 
\end{document}